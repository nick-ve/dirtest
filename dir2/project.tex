\documentclass[11pt,a4paper,pdftex]{report}
\usepackage{a4ws}
\usepackage{nvecmd}

\begin{document}

\begin{center}
{\bf Neutrino astronomie met IceCube en zijn DeepCore uitbreiding}
\end{center}

Neutrino's zijn bijzondere astronomische boodschappers; zij zijn de enige deeltjes die informatie naar
de aarde kunnen overbrengen van kosmische verschijnselen die zich aan de rand van het Universum afspelen.\\
Met de ingebruikname van IceCube, het grootste neutrino observatorium ter wereld op de Zuidpool,
is een wereldwijde zoektocht ingezet naar hoge-energie neutrino's afkomstig van kosmische fenomenen.

Kosmische Gamma Flitsen en uitbarstingen van Actieve Galactische Kernen zijn de meest krachtige explosieve
fenomenen die wij tot op heden in het heelal hebben waargenomen. Tevens zijn zij ook de minst begrepen
verschijnselen.
Op basis van de huidige beschikbare gegevens menen wij dat het hier processen betreft in relatie met
zwarte gaten, waarbij Gamma Flitsen zelfs het ontstaan van zwarte gaten zouden kunnen reflecteren.\\
Indien bovenstaande visie juist is, dan verwacht men naast de reeds waargenomen elektromagnetische signalen
ook een hadronische component in de emissie. 
Observaties van hoog-energetische fotonen geven hierover geen eenduidige conclusies, maar de detectie van
geassocieerde neutrino emissie zou eenduidig wijzen op een hadronische component.
Bovendien reflecteert de observatie van een tijdsverschil tussen de elektromagnetische en de neutrino burst
verschillende productie- en versnellingsmechanismen, wat inzicht schept in de fysische processen die ten
grondslag liggen aan deze gebeurtenissen.\\
Het aantonen en in detail bestuderen van hoog-energetische neutrino's afkomstig van bovengenoemde kosmische
explosies vormt dan ook het hoofdthema van het hier voorgestelde onderzoek.

De combinatie van IceCube gegevens met elektromagnetische observaties opent voor het eerst in de
geschiedenis de mogelijkheid om alle bovengenoemde aspecten simultaan te bestuderen.
Nu de FERMI sateliet operationeel is en IceCube reeds voor meer dan 90\% is geinstalleerd en metingen
verricht, is de tijd rijp om onderzoek in deze richting te starten.\\
Daar waar FERMI de elektromagnetische informatie verschaft, zullen IceCube en zijn DeepCore uitbreiding ons
in staat stellen de zoektocht naar gerelateerde neutrino activiteit uit te voeren met ongekende gevoeligheid.\\
DeepCore is een dicht rooster sensoren geplaatst op de grootst mogelijke diepte in het Antarctisch ijs,
volledig omgeven door de standaard IceCube sensoren die als veto gebruikt kunnen worden.
Deze unieke eigenschappen van DeepCore leveren een omgeving met zeer lage achtergrond en laten toe het
energiebereik van IceCube naar beneden uit te breiden tot 20 GeV, wat resulteert in een significante verhoging
van de gevoeligheid en een uitbreiding tot een $4\pi$ ruimtehoekbereik, inclusief de Zuidelijke hemel.
Bijgevolg zal de DeepCore uitbreiding van IceCube ons tevens toelaten bronnen in onze Melkweg te bestuderen,
inclusief het massief zwart gat in het centrum ervan.\\
De DeepCore component van IceCube is in december 2009 geinstalleerd en de complete detector zal in maart 2010
in bedrijf worden genomen.

Het onderzoek van dr. Singh zal zich richten op het gebruik van IceCube's DeepCore component
om daarmee neutrino's afkomstig van kosmische Gamma Flitsen en uitbarstingen van Actieve Galactische Kernen
met de grootste gevoeligheid te detecteren.\\
Schematisch weergegeven zullen de werkzaamheden bestaan uit~:
%
\begin{itemize}
\item Het evalueren en kalibreren van de verkregen detector signalen.
\item Optimalisatie van de IceCube en DeepCore reconstructie algorithmen.
\item Ontwikkelen van selectiecriteria om de benodigde meetgegevens te verkrijgen.
\item Selectie van de relevante meetgegevens voor bovengenoemde studies.
\item Het uitvoeren van de data analyse en (astro)fysische interpretatie.
\item Verslaglegging van de behaalde resultaten middels wetenschappelijke publicaties.
\end{itemize}

\end{document}
